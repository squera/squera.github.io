%%%%%%%%%%%%%%%%%%%%%%%%%%%%%%%%%%%%%%%%%%%%%%%%%%%%%%%%%%%%%%%%%%%%%%%%%%%%%%%%%%%%%%%%%%%%%%%%%%%%%%%%%%%%%%%%%%%%%%%%%%%%%%%%%%%%%%%%%%%%%%%%%%%%%%%%%%%%%%%%%%%%%%%%%%%%%%%%%
%%%%%%%%%%%%%%%%%%%%%%%%%%%%%%%%%%%%%%%%%%%%%%%%%%%%%%%%%%%							COMMANDS						  %%%%%%%%%%%%%%%%%%%%%%%%%%%%%%%%%%%%%%%%%%%%%%%%%%%%%%%%%%%
%%%%%%%%%%%%%%%%%%%%%%%%%%%%%%%%%%%%%%%%%%%%%%%%%%%%%%%%%%%%%%%%%%%%%%%%%%%%%%%%%%%%%%%%%%%%%%%%%%%%%%%%%%%%%%%%%%%%%%%%%%%%%%%%%%%%%%%%%%%%%%%%%%%%%%%%%%%%%%%%%%%%%%%%%%%%%%%%%

%%%%%%%%%%%%%%%%%%%%%%%%%%%%%%%%%%%%%%%%%%%%%%%%%%%%%%%%%%%
%	TODO annotations
%%%%%%%%%%%%%%%%%%%%%%%%%%%%%%%%%%%%%%%%%%%%%%%%%%%%%%%%%%%
\newcommand{\MP}[1]{\todo[color=blue!30]{MP TODO: #1}}
\newcommand{\MPin}[1]{\todo[color=blue!30,inline]{MP TODO: #1}}
\newcommand{\DD}[1]{\todo[color=yellow!30]{DS TODO: #1}}

%%%%%%%%%%%%%%%%%%%%%%%%%%%%%%%%%%%%%%%%%%%%%%%%%%%%%%%%%%%
%	Math formatting
%%%%%%%%%%%%%%%%%%%%%%%%%%%%%%%%%%%%%%%%%%%%%%%%%%%%%%%%%%%
\newcommand{\mi}[1]{\ensuremath{\mathit{#1}}}
\newcommand{\mr}[1]{\ensuremath{\mathrm{#1}}}
\newcommand{\mt}[1]{\ensuremath{\texttt{#1}}}
\newcommand{\mtt}[1]{\ensuremath{\mathtt{#1}}}
\newcommand{\mf}[1]{\ensuremath{\mathbf{#1}}}
\newcommand{\mk}[1]{\ensuremath{\mathfrak{#1}}}
\newcommand{\mc}[1]{\ensuremath{\mathcal{#1}}}
\newcommand{\ms}[1]{\ensuremath{\mathsf{#1}}}
\newcommand{\mb}[1]{\ensuremath{\mathbb{#1}}}
\newcommand{\msc}[1]{\ensuremath{\mathscr{#1}}}

\newcommand{\isdef}[0]{\ensuremath{\mathrel{\overset{\makebox[0pt]{\mbox{\normalfont\tiny\sffamily def}}}{=}}}}

\newcommand{\lift}[1]{\ensuremath{\lceil#1\rceil}}

% http://tex.stackexchange.com/questions/5502/how-to-get-a-mid-binary-relation-that-grows
\newcommand{\relmiddle}[1]{\mathrel{}\middle#1\mathrel{}}

\DeclareMathOperator\mydefsym{\ensuremath{\iangleq}}
\newcommand\bnfdef{\ensuremath{\mathrel{::=}}}

%%%%%%%%%%%%%%%%%%%%%%%%%%%%%%%%%%%%%%%%%%%%%%%%%%%%%%%%%%%
%	Math shortcuts
%%%%%%%%%%%%%%%%%%%%%%%%%%%%%%%%%%%%%%%%%%%%%%%%%%%%%%%%%%%
\newcommand{\OB}[1]{\ensuremath{\overline{#1}}}
\newcommand{\llb}{\llbracket}
\newcommand{\rrb}{\rrbracket}
\newcommand{\lla}{\mathopen{\ll}}
\newcommand{\rra}{\mathclose{\gg}}
\newcommand{\ra}{\rightarrow}
\newcommand{\Ra}{\Rightarrow}
\newcommand{\la}{\leftarrow}
\newcommand{\La}{\Leftarrow}
\newcommand{\Da}[1]{\ensuremath{\Downarrow^{#1}}}
\newcommand{\Ua}[1]{\ensuremath{\Downarrow^{#1}}}

\newcommand{\myset}[2]{\ensuremath{\left\{#1 ~\relmiddle|~ #2\right\}}}
\newcommand{\partof}[1]{\ensuremath{\mc{P}(#1)}}

\newcommand{\divr}[0]{\ensuremath{\!\!\Uparrow}\xspace}
\newcommand{\ndivr}[0]{\ensuremath{\!\!\not\Uparrow}\xspace}
\newcommand{\divrs}[0]{\ensuremath{\src{\Uparrow}}\xspace}
\newcommand{\divrt}[0]{\ensuremath{\trg{\Uparrow}}\xspace}
\newcommand{\divrc}[0]{\ensuremath{\com{\Uparrow}}\xspace}

\newcommand{\term}[0]{\ensuremath{\!\!\Downarrow}\xspace}
\newcommand{\termsl}[0]{\ensuremath{\src{\Downarrow}}\xspace}
\newcommand{\termt}[0]{\ensuremath{\trg{\Downarrow}}\xspace}
\newcommand{\termc}[0]{\ensuremath{\com{\Downarrow}}\xspace}


\newcommand{\traces}[3]{\ensuremath{{TR}^{#2}_{#3}(#1)}}
\newcommand{\trs}[1]{\src{\traces{#1}{}{}}}
\newcommand{\trt}[1]{\trg{\traces{#1}{}{}}}
\newcommand{\trc}[1]{\com{\traces{#1}{}{}}}

%%%%%%%%%%%%%%%%%%%%%%%%%%%%%%%%%%%%%%%%%%%%%%%%%%%%%%%%%%%
%	Math environments
%%%%%%%%%%%%%%%%%%%%%%%%%%%%%%%%%%%%%%%%%%%%%%%%%%%%%%%%%%%
\newcommand*{\QEDA}{\hfill\ensuremath{\blacksquare}}%

\AtEndEnvironment{problem}{\null\hfill\QEDA}
\AtEndEnvironment{example}{}%\null\hfill$\boxdot$}
%\AtEndEnvironment{proof}{\null\hfill\qed}

\Crefname{lstlisting}{Listing}{Listings}
\Crefname{problem}{Problem}{Problems}

\Crefname{equation}{Rule}{Rules}

\newenvironment{proofsketch}{\trivlist\item[]\emph{Proof Sketch}.\xspace}{\unskip\nobreak\hskip 1em plus 1fil\nobreak$\Box$\parfillskip=0pt\endtrivlist}
%%%%%%%%%%%%%%%%%%%%%%%%%%%%%%%%%%%%%%%%%%%%%%%%%%%%%%%%%%%
%	Compiler 
%%%%%%%%%%%%%%%%%%%%%%%%%%%%%%%%%%%%%%%%%%%%%%%%%%%%%%%%%%%
\newcommand{\genlang}[1]{\ensuremath{\lambda^{#1}}}
\newcommand{\stlcf}[0]{\src{\genlang{fx}}}
\newcommand{\stlcm}[0]{\trg{\genlang{\mu}}}

\newcommand{\genlangF}[1]{\ensuremath{F^{#1}}}
\newcommand{\sysff}[0]{\src{\genlangF{fx}}}
\newcommand{\sysfm}[0]{\trg{\genlangF{\mu}}}

\newcommand{\compskel}[3]{\ensuremath{\bl{\left\llbracket \src{#1} \right\rrbracket^{#2}_{#3}}}}
\newcommand{\comp}[1]{\compskel{\bl{#1}}{}{}}
\newcommand{\compgen}[1]{\compskel{#1}{\S}{\T}}
\newcommand{\compstlc}[1]{\compskel{#1}{\stlcf}{\stlcm}}

\newcommand{\compst}[1]{\compskel{#1}{\So}{\To}}
\newcommand{\compsr}[1]{\compskel{#1}{\So}{\Tr}}
\newcommand{\comph}[1]{\compskel{#1}{\Sh}{\Th}}
\newcommand{\compcaps}[1]{\compskel{#1}{\Sh}{\Tcaps}}
\newcommand{\compsgx}[1]{\compskel{#1}{\Sh}{\Tsgx}}
\newcommand{\compm}[1]{\compskel{#1}{\Sm}{\Tm}}

\newcommand{\funname}[1]{\mtt{#1}}
\newcommand{\fun}[2]{\ensuremath{{\bl{\funname{#1}\left(#2\right)}}}\xspace}
\newcommand{\dom}[1]{\fun{dom}{#1}}

% \newcommand{\Nat}[0]{\ensuremath{\mb{N}}\xspace}

\newcommand{\backtrskel}[3]{\ensuremath{\bl{\left\langle\!\left\langle \trg{#1} \right\rangle\!\right\rangle^{#2}_{#3}}}}
\newcommand{\backtrskeltr}[3]{\ensuremath{\bl{\left\langle\!\left\langle\!\left\langle \trg{#1} \right\rangle\!\right\rangle\!\right\rangle^{#2}_{#3}}}}
\newcommand{\backtr}[1]{\backtrskel{#1}{}{}}
\newcommand{\backtrgen}[1]{\backtrskel{#1}{\T}{\S}}
\newcommand{\backtrstlc}[1]{\backtrskel{#1}{\stlcm}{\stlcf}}

\newcommand{\backtrts}[1]{\backtrskel{#1}{\To}{\So}}
\newcommand{\backtrtstr}[1]{\backtrskeltr{#1}{\Tr}{\So}}
\newcommand{\backtrcaps}[1]{\backtrskeltr{#1}{\Tcaps}{\Sh}}
\newcommand{\backtrsgx}[1]{\backtrskeltr{#1}{\Tsgx}{\Sh}}
\newcommand{\backtrm}[1]{\backtrskeltr{#1}{\Tr}{\So}}
%%%%%%%%%%%%%%%%%%%%%%%%%%%%%%%%%%%%%%%%%%%%%%%%%%%%%%%%%%%
%	Language shortcuts
%%%%%%%%%%%%%%%%%%%%%%%%%%%%%%%%%%%%%%%%%%%%%%%%%%%%%%%%%%%
\renewcommand{\S}[0]{\sgen{w}{}}
\newcommand{\T}[0]{\tgen{w}{}}

\newcommand{\sgen}[2]{\src{S^{#1}_{#2}}\xspace}
\newcommand{\tgen}[2]{\trg{T^{#1}_{#2}}\xspace}

\newcommand{\So}[0]{\sgen{}{\tau}}
\newcommand{\To}[0]{\tgen{}{u}}
\newcommand{\Tr}[0]{\tgen{}{r}}
\newcommand{\Sh}[0]{\sgen{}{\ell}}
\newcommand{\Th}[0]{\tgen{}{n}}
\newcommand{\Tcaps}[0]{\tgen{}{k}}
\newcommand{\Tsgx}[0]{\tgen{}{e}}

\newcommand{\Sm}[0]{\sgen{}{\tau}}
\newcommand{\Tm}[0]{\tgen{}{u}}


\newcommand{\ctx}[0]{\ensuremath{C}}%{\mk{C}}}
\newcommand{\ctxs}[0]{\src{\ctx}\xspace}
\newcommand{\ctxt}[0]{\trg{\ctx}\xspace}%M
\newcommand{\ctxc}[0]{\com{\ctx}\xspace}%P
\newcommand{\ctxh}[1]{\ctx\hole{#1}}
\newcommand{\ctxhs}[1]{\ctxs\src{\hole{#1}}\xspace}
\newcommand{\ctxht}[1]{\ctxt\trg{\hole{#1}}\xspace}%M
\newcommand{\ctxhc}[1]{\ctxc\com{\hole{#1}}\xspace}%P
\newcommand{\hole}[1]{\ensuremath{\left[#1\right]}}

\newcommand{\evalctx}[0]{\ensuremath{\mb{E}}}
\newcommand{\evalctxs}[0]{\src{\evalctx}\xspace}
\newcommand{\evalctxt}[0]{\trg{\evalctx}\xspace}
\newcommand{\evalctxc}[0]{\com{\evalctx}\xspace}
\newcommand{\evalctxhs}[1]{\src{\evalctx\hole{#1}}\xspace}
\newcommand{\evalctxht}[1]{\trg{\evalctx\hole{#1}}\xspace}
\newcommand{\evalctxhc}[1]{\com{\evalctx\hole{#1}}\xspace}

\newcommand{\behav}[1]{\fun{Behav}{#1}}
\newcommand{\sem}{\boldsymbol{\rightsquigarrow}}



\newcommand{\Bool}[0]{{{Bool}}\xspace}
\newcommand{\Nat}[0]{{{Nat}}\xspace}
\newcommand{\Bools}[0]{\src{{Bool}}\xspace}
\newcommand{\Nats}[0]{\src{{Nat}}\xspace}
\newcommand{\Boolt}[0]{\trg{{Bool}}\xspace}
\newcommand{\Natt}[0]{\trg{{Nat}}\xspace}
\newcommand{\Ints}[0]{\src{{Int}}\xspace}
\newcommand{\Intt}[0]{\trg{{Int}}\xspace}
\newcommand{\Units}[0]{\src{{Unit}}\xspace}
\newcommand{\Unitt}[0]{\trg{{Unit}}\xspace}
\newcommand{\trues}[0]{\src{{true}}\xspace}
\newcommand{\falses}[0]{\src{{false}}\xspace}
\newcommand{\units}[0]{\src{{unit}}\xspace}
\newcommand{\truet}[0]{\trg{{true}}\xspace}
\newcommand{\falset}[0]{\trg{false}\xspace}
\newcommand{\unitt}[0]{\trg{unit}\xspace}

\newcommand{\truev}[0]{{{true}}\xspace}
\newcommand{\falsev}[0]{{{false}}\xspace}
\newcommand{\unitv}[0]{{{unit}}\xspace}


\newcommand{\srce}[0]{\src{\emptyset}\xspace}
\newcommand{\trge}[0]{\trg{\emptyset}\xspace}
\newcommand{\come}[0]{\com{\emptyset}\xspace}

\newcommand{\SInit}[1]{\ensuremath{{\Omega_0}({#1})}\xspace}
\newcommand{\SInits}[1]{\ensuremath{\src{\Omega_0}(\src{#1})}\xspace}
\newcommand{\SInitt}[1]{\ensuremath{\trg{\Omega_0}(\trg{#1})}\xspace}

%%%%%%%%%%%%%%%%%%%%%%%%%%%%%%%%%%%%%%%%%%%%%%%%%%%%%%%%%%%
%	Language formatting
%%%%%%%%%%%%%%%%%%%%%%%%%%%%%%%%%%%%%%%%%%%%%%%%%%%%%%%%%%%
\newcommand{\neutcol}[0]{black}
\newcommand{\stlccol}[0]{RoyalBlue}
\newcommand{\ulccol}[0]{RedOrange}
\newcommand{\commoncol}[0]{black}    % CarnationPink

\newcommand{\col}[2]{\ensuremath{{\color{#1}{#2}}}}

\newcommand{\src}[1]{\ms{\col{\stlccol}{#1}}}
\newcommand{\trgb}[1]{\mf{\col{\ulccol }{#1}}} 
\newcommand{\trg}[1]{{\mf{\col{\ulccol }{#1}}}}
% MARCO: \bm is notorious to break things around. it's there only to make bold math letters. we can remove it if necessary.
% it is currently removed -- the paretheses are still there though -- as it did go beyond its scope, i did not know how to remove it (\mr did nont work)
% it was affecting stuff inside the compilation brackets, making source stuff bold ... 
%if we know of a solution, we can add \bm at the beginning here and the bold-removal command in the core of \compgen
\newcommand{\bl}[1]{\col{\neutcol }{#1}}
\newcommand{\com}[1]{\mi{\col{\commoncol }{#1}}}

\newcommand{\hil}[1]{\colorbox{yellow}{#1}}

%%%%%%%%%%%%%%%%%%%%%%%%%%%%%%%%%%%%%%%%%%%%%%%%%%%%%%%%%%%
%	Type rules
%%%%%%%%%%%%%%%%%%%%%%%%%%%%%%%%%%%%%%%%%%%%%%%%%%%%%%%%%%%
\newcounter{typerule}
\crefname{typerule}{rule}{rules}

\newcommand{\typeruleInt}[5]{%									    % #1 is the title, #2 is the hypotheses. #3 is the thesis, #4 is the label for referencing
	\def\thetyperule{#1}%
	\refstepcounter{typerule}%
	\label{tr:#4}%
	%\ensuremath{\begin{array}{c}\textsf{\scriptsize ({#1})} \\#2 \\\hline{\ensuremath{#3}}\end{array}} \inference
  \ensuremath{\begin{array}{c}#5 \inference{#2}{#3}\end{array}} 
}
\newcommand{\typerule}[4]{%									        % #1 is the title, #2 is the hypotheses. #3 is the thesis, #4 is the label for referencing
  \typeruleInt{#1}{#2}{#3}{#4}{\textsf{\scriptsize ({#1})} \\      }
}
\newcommand{\typerulenolabel}[4]{%									% #1 is the title, #2 is the hypotheses. #3 is the thesis, #4 is the label for referencing
  \typeruleInt{#1}{#2}{#3}{#4}{}
}

%%%%%%%%%%%%%%%%%%%%%%%%%%%%%%%%%%%%%%%%%%%%%%%%%%%%%%%%%%%
%	Figures, tables and tikz
%%%%%%%%%%%%%%%%%%%%%%%%%%%%%%%%%%%%%%%%%%%%%%%%%%%%%%%%%%%
\pgfdeclarelayer{background}
\pgfdeclarelayer{veryback}
\pgfdeclarelayer{veryback2}
\pgfdeclarelayer{veryback3}
\pgfdeclarelayer{back2}
\pgfdeclarelayer{foreground}
\pgfdeclarelayer{foreground2}
\pgfdeclarelayer{foreground3}
\pgfdeclarelayer{foreground4}
\pgfsetlayers{veryback3,veryback2,veryback,background,back2,main,foreground,foreground2,foreground3,foreground4}

\newcommand{\tikzpic}[1]{
\begin{tikzpicture}[shorten >=1pt,auto,node distance=6mm]
\tikzstyle{state} =[fill=white,minimum size=4pt]
\tikzstyle{field} =[fill=gray!5,draw=black!70, rectangle, minimum width={width("whiskersfieldww")+2pt}]]
#1
\end{tikzpicture}
}

\newcommand{\tikzpicT}[1]{
\begin{tikzpicture}[shorten >=1pt,node distance=12mm,auto]
\tikzstyle{state} =[fill=white,minimum size=4pt]
#1
\end{tikzpicture}
}

\newcommand{\myfig}[3]{\begin{figure} [!h]
#1
\caption{\label{fig:#2}#3}
\end{figure}}

\newcommand{\mytab}[3]{\begin{table} [!h]
\centering
#1

\caption{\label{tab:#2}#3}
\end{table}}

%%%%%%%%%%%%%%%%%%%%%%%%%%%%%%%%%%%%%%%%%%%%%%%%%%%%%%%%%%%
% Misc
%%%%%%%%%%%%%%%%%%%%%%%%%%%%%%%%%%%%%%%%%%%%%%%%%%%%%%%%%%%
\newcommand{\etal}[0]{\textit{et al.}\xspace} 

\newcommand{\qed}[0]{\hfill$\square$}

\newcommand{\BREAK}[0]{
\botrule
\begin{center}$\spadesuit$\end{center}
\botrule}

\newcommand{\mytoprule}[1]{\vspace{1mm}\noindent\hrulefill\ \raisebox{-0.5ex}{\fbox{\ensuremath{#1}}} \hrulefill\hrulefill\hrulefill\vspace{0.5mm}}

\def\botrule{\vspace{0mm}\hrule\vspace{2mm}}

\newcommand{\link}[1]{\href{#1}{#1}}

\newcommand{\myparagraph}[1]{ \smallskip \noindent\noindent\textit{#1}~}

\DeclareMathOperator\compat{\ensuremath{\raisebox{1mm}{$\frown$}}}

%%%%%%%%%%%%%%%%%%%%%%%%%%%%%%%%%%%%%%%%%%%%%%%%%%%%%%%%%%%
% Listings
%%%%%%%%%%%%%%%%%%%%%%%%%%%%%%%%%%%%%%%%%%%%%%%%%%%%%%%%%%%
\newcommand{\diff}[1]{\color{red!60}{\lst{#1}}}

\definecolor{mygreen}{rgb}{0,0.6,0}
\definecolor{mygray}{rgb}{0.5,0.5,0.5}
\definecolor{mymauve}{rgb}{0.58,0,0.82}

\newcommand{\lstb}[1]{\ensuremath{\text{\texttt{\textbf{#1}}}}}
\newcommand{\lst}[1]{\ensuremath{\text{\texttt{#1}}}}

\lstdefinelanguage{Java} %thanks to whoever did this
{morekeywords={abstract, all, and, as, assert, but, check, disj, else, exactly, extends, fact, for, fun, iden, if, iff, implies, in, Int, int, let, lone, module, no, none, not, one, open, or, part, pred, run, seq, set, sig, some, sum, then, univ, package, class, public, private, null, return, new, interface, extern, object, implements, System, static, super, try , catch, throw, throws, Unit, var, val, of, principal, trust},
sensitive=true,
%commentstyle=\color{mygreen},
%morecomment=[l][\small\itshape\color{mygreen}]{--},
%morecomment=[l][\small\itshape\color{mygreen}]{//},
%morecomment=[s][\small\itshape\color{mygreen}]{/*}{*/},
%keywordstyle=\color{blue!60!black}\bfseries,
%stringstyle=\ttfamily\color{red!50!brown},
%identifierstyle=\color{purple!40!black},
keywordstyle=\bfseries\color{green!40!black},
commentstyle=\itshape\color{purple!40!black},
morecomment=[l][\small\itshape\color{purple!40!black}]{//},
identifierstyle=\color{blue},
stringstyle=\color{orange},
basicstyle=\small,
basicstyle={\small\ttfamily},
numbers=left,
numberstyle=\tiny\color{mygray},
tabsize=2,
numbersep=3pt,
breaklines=true,
lineskip=-2pt,
stepnumber=1,
captionpos=b,
breaklines=true,
breakatwhitespace=false,
showspaces=false,
showtabs=false,
float=!h,
%frame=single,
columns=fullflexible,escapeinside={(*@}{@*)},
moredelim=**[is][\color{red!60}]{@}{@},
literate={->}{{$\to$}}1 {^}{{$\mspace{-3mu}\widehat{\quad}\mspace{-3mu}$}}1
{<}{$<$ }2 {>}{$>$ }2 {>=}{$\geq$ }2 {=<}{$\leq$ }2
{<:}{{$<\mspace{-3mu}:$}}2 {:>}{{$:\mspace{-3mu}>$}}2
{=>}{{$\Rightarrow$ }}2 {+}{$+$ }2 {++}{{$+\mspace{-8mu}+$ }}2
{<=>}{{$\Leftrightarrow$ }}2 {+}{$+$ }2 {++}{{$+\mspace{-8mu}+$ }}2
{\~}{{$\mspace{-3mu}\widetilde{\quad}\mspace{-3mu}$}}1
{!=}{$\neq$ }2 {*}{${}^{\ast}$}1 %{.}{$\cdot$}1
{\#}{$\#$}1
}
\lstset{language=Java,numbersep=5pt,frame=single}

%%%%%%%%%%%%%%%%%%%%%%%%%%%%%%%%%%%%%%%%%%%%%%%%%%%%%%%%%%%
%	Contextual equivalence
%%%%%%%%%%%%%%%%%%%%%%%%%%%%%%%%%%%%%%%%%%%%%%%%%%%%%%%%%%%
\DeclareMathOperator\niff{\ensuremath{\nLeftrightarrow}}

\DeclareMathOperator\ceq{\ensuremath{\mathrel{\simeq_{\mi{ctx}}}}}
\DeclareMathOperator\nceq{\mathrel{\not\simeq_{\mi{ctx}}}}

\DeclareMathOperator\ceqs{\src{\ceq}}
\DeclareMathOperator\ceqt{\trg{\ceq}}
\DeclareMathOperator\ceqc{\com{\ceq}}

\DeclareMathOperator\nceqs{\src{\nceq}}
\DeclareMathOperator\nceqt{\trg{\nceq}}

% cross language stuff
\newcommand\subsetsim{\mathrel{\substack{\textstyle\subset\\[-0.2ex]\textstyle\sim}}}
\newcommand\supsetsim{\mathrel{\substack{\textstyle\supset\\[-0.2ex]\textstyle\sim}}}
\newcommand\sqsubsetsim{\mathrel{\substack{\textstyle\sqsubset\\[-0.2ex]\textstyle\sim}}}

\newcommand{\divt}[0]{\trg{\uparrow}\xspace}
\newcommand{\tert}[0]{\trg{\downarrow}\xspace}
\newcommand{\divs}[0]{\src{\uparrow}\xspace}
\newcommand{\ters}[0]{\src{\downarrow}\xspace}
\newcommand{\divc}[0]{\com{\uparrow}\xspace}
\newcommand{\terc}[0]{\com{\downarrow}\xspace}

%%%%%%%%%%%%%%%%%%%%%%%%%%%%%%%%%%%%%%%%%%%%%%%%%%%%%%%%%%%
% Sets envs
%%%%%%%%%%%%%%%%%%%%%%%%%%%%%%%%%%%%%%%%%%%%%%%%%%%%%%%%%%%
\newcommand{\set}[1]{\ensuremath{\widehat{#1}} }%\setb{#1}}

\newcommand{\card}[1]{\ensuremath{|\!|{#1}|\!|}}

%%%%%%%%%%%%%%%%%%%%%%%%%%%%%%%%%%%%%%%%%%%%%%%%%%%%%%%%%%%
% Missing envs
%%%%%%%%%%%%%%%%%%%%%%%%%%%%%%%%%%%%%%%%%%%%%%%%%%%%%%%%%%%
\theoremstyle{definition}
\newtheorem{assumption}{Assumption}
\newtheorem{notation}{Notation}
% \newtheorem{definition}{Definition}
% \newtheorem{theorem}{Theorem}
% \newtheorem{lemma}{Lemma}
\newtheorem{property}{Property}
% \newtheorem{example}{Example}
\newtheorem{informal}{Informal definition}
% \newtheorem{corollary}{Corollary}
\Crefname{corollary}{Corollary}{Corollaries}
\Crefname{informal}{Definition}{Definition}
\Crefname{assumption}{Assumption}{Assumptions}
\crefname{assumption}{Assumption}{Assumptions}
\Crefname{property}{Property}{Properties}
\crefname{property}{Property}{Properties}


%%%%%%%%%%%%%%%%%%%%%%%%%%%%%%%%%%%%%%%%%%%%%%%%%%%%%%%%%%%
% Lambda 
%%%%%%%%%%%%%%%%%%%%%%%%%%%%%%%%%%%%%%%%%%%%%%%%%%%%%%%%%%%
\newcommand{\has}[2]{\ensuremath{#1~has~#2}}
\newcommand{\lam}[2]{\ensuremath{\lambda #1\ldotp #2}}
\newcommand{\pair}[1]{\ensuremath{\left\langle#1\right\rangle}}
\newcommand{\projone}[1]{\ensuremath{#1.1}}
\newcommand{\projtwo}[1]{\ensuremath{#1.2}}
\newcommand{\fix}[1]{\ensuremath{fix}_{#1}}
\newcommand{\case}{\ensuremath{{case}}}
\newcommand{\of}{\ensuremath{{of}}}
\newcommand{\caseof}[3]{\ensuremath{{case}~#1~{of}~\inl{x_1}\mapsto #2\mid\inr{x_2}\mapsto #3}}
\newcommand{\caseofmultlin}[3]{{{case}~#1~{of}~\inl{x_1}\mapsto #2\mid\inr{x_2}\mapsto #3}}
\newcommand{\casefoldeds}[3]{
	\src{\case}~#1~\src{\of}\left|\begin{aligned}
			&
			\src{\inl{x_1}\mapsto} #2
			\\
			&
			\src{\inr{x_2}\mapsto} #3
		\end{aligned}\right.
}
\newcommand{\casefoldedt}[3]{
	\trg{\case}~#1~\trg{\of}\left|\begin{aligned}
			&
			\trg{\inl{x_1}\mapsto} #2
			\\
			&
			\trg{\inr{x_2}\mapsto} #3
		\end{aligned}\right.
}
\newcommand{\ifte}[3]{\ensuremath{{if}~#1~{then}~#2~{else}~#3}}
\newcommand{\iftes}[3]{\ensuremath{\src{if}~#1~\src{then}~#2~\src{else}~#3}}
\newcommand{\iftet}[3]{\ensuremath{\trg{if}~#1~\trg{then}~#2~\trg{else}~#3}}

\newcommand{\inl}[1]{\ensuremath{{inl}~#1}}
\newcommand{\inr}[1]{\ensuremath{{inr}~#1}}
\newcommand{\wrong}[0]{\trg{wrong}}
\newcommand{\protd}[1]{\ensuremath{\funname{\trg{protect}_{\src{#1}}}}}
\newcommand{\confi}[1]{\ensuremath{\funname{\trg{confine}_{\src{#1}}}}}
\newcommand{\conf}[1]{\confi{#1}}
\newcommand{\fold}[1]{\ensuremath{{fold}_{#1}}}
\newcommand{\unfold}[1]{\ensuremath{{unfold}_{#1}}}

\newcommand{\letin}[3]{{let}~#1=#2~{in}~#3}
\newcommand{\letins}[3]{\src{let}~#1\src{=}#2~\src{in}~#3}
\newcommand{\letint}[3]{\trg{let}~#1\trg{=}#2~\trg{in}~#3}
\newcommand{\mylet}[2]{\mr{let}~#1=#2}
\newcommand{\call}[1]{\mr{call}~#1}
\newcommand{\ret}[0]{\mr{ret}}

\newcommand{\print}[1]{print~#1}

\newcommand{\letcall}[3]{{let}~#1{=}#2~{in~call}~#3}

\newcommand{\letnew}[3]{{let}~#1={new}~#2~{in}~#3}
\newcommand{\letread}[3]{{let}~#1={read}~#2~{in}~#3}
\newcommand{\letreadnorm}[3]{\trg{let}~#1=\trg{read}~#2~\trg{in}~#3}
\newcommand{\letwrite}[4]{{let}~#1={write}~#2~{at}~#3~{in}~#4}
\newcommand{\letwritenorm}[4]{\trg{let}~#1=\trg{write}~#2~\trg{at}~#3~\trg{in}~#4}

\newcommand{\letnews}[3]{\src{let}~#1\src{=new}~#2~\src{in}~#3}
\newcommand{\letreads}[3]{\src{let}~#1\src{=read}~#2~\src{in}~#3}
\newcommand{\letwrites}[4]{\src{let}~#1\src{=write}~#2~\src{at}~#3~\src{in}~#4}

\newcommand{\letnewt}[3]{\trg{let}~#1\trg{=new}~#2~\trg{in}~#3}
\newcommand{\letreadt}[4]{\trg{let}~#1\trg{=read}~#2~\trg{with}~#3~\trg{in}~#4}
\newcommand{\letwritet}[5]{\trg{let}~#1\trg{=write}~#2~\trg{at}~#3~\trg{with}~#4~\trg{in}~#5}
\newcommand{\lethidet}[3]{\trg{let}~#1\trg{=hide}~#2~\trg{in}~#3}

\newcommand{\mybegin}[0]{\mr{beg}~}
\newcommand{\myend}[0]{\mr{end}}
\newcommand{\myskip}[0]{skip}
\newcommand{\myskips}[0]{\src{\skip}}
\newcommand{\myskipt}[0]{\trg{\skip}}

\newcommand{\refty}[1]{\src{Ref~#1}}

\newcommand{\type}[3]{\ensuremath{ \left\{#1:#2\relmiddle|#3 \right\}}}

\newcommand{\matgen}[2]{\ensuremath{\mu #1\ldotp#2}}
\newcommand{\mat}[0]{\matgen{\alpha}{\tau}}

\newcommand{\erase}[1]{\fun{erase}{\src{#1}}}

\newcommand{\fail}[0]{{fail}}
\newcommand{\fails}[0]{\src{fail}\xspace}
\newcommand{\failt}[0]{\trg{fail}\xspace}
\newcommand{\failc}[0]{\com{fail}\xspace}

\newcommand{\fatsem}[0]{\formatCompilers{fat}}

\newcommand{\redgen}[1]{\ensuremath{ \hookrightarrow^{#1} }}
\newcommand{\nredgen}[1]{\ensuremath{\not\hookrightarrow^{#1}}}
\newcommand{\red}[0]{\redgen{}}
\newcommand{\redp}[0]{\redgen{0}}
\newcommand{\nred}[0]{\nredgen{}}
\newcommand{\redstar}[0]{\redgen{*}}

\newcommand{\nredt}[0]{\trg{\nred}}
\newcommand{\nreds}[0]{\src{\nred}}
\newcommand{\nredc}[0]{\com{\nred}}
\newcommand{\redstars}[0]{\src{\redstar}}
\newcommand{\redstarc}[0]{\com{\redstar}}
\newcommand{\redstart}[0]{\trg{\redstar}}

\DeclareMathOperator\reds{\src{\red}}
\DeclareMathOperator\redt{\trg{\red}}
\DeclareMathOperator\redc{\com{\red}}

\newcommand{\diverge}{\Uparrow}

\AtEndEnvironment{example}{\null\hfill$\boxdot$}

\makeatletter
\xdef\@thefnmark{\@empty}
\newcommand{\thmref}[1]{\cref{#1}~(\nameref{#1})}
\newcommand{\Thmref}[1]{\Cref{#1}~(\nameref{#1})}
\makeatother

\newcommand{\subst}[2]{\ensuremath{\bl{\left[#1\relmiddle/#2\right]}}} %replace 1 in place of 2
\newcommand{\subs}[2]{\subst{\src{#1}}{\src{#2}}}
\newcommand{\subt}[2]{\subst{\trg{#1}}{\trg{#2}}}



%%%%%%%%%%%%%%%%%%%%%%%%%%%%%%%%%%%%%%%%%%%%%%%%%%%%%%%%%%%
%	naming HP and referring to them
%%%%%%%%%%%%%%%%%%%%%%%%%%%%%%%%%%%%%%%%%%%%%%%%%%%%%%%%%%%
\newcounter{hps}
\crefname{hps}{}{}
\newcommand{\hpdef}[2]{ % nome, label
	\def\thehps{#1}%
  	\refstepcounter{hps}%
  	\label{hp:#2}%
  	#1
}
\newcommand{\hpref}[1]{\Cref{hp:#1}}




%%%%%%%%%%%%%%%%%%%%%%%%%%%%%%%%%%%%%%%%%%%%%%%%%%%%%%%%%%%
%	misc
%%%%%%%%%%%%%%%%%%%%%%%%%%%%%%%%%%%%%%%%%%%%%%%%%%%%%%%%%%%
\renewcommand{\emptyset}[0]{\varnothing}
\newcommand{\changetargetcolour}[0]{\renewcommand{\ulccol}[0]{Magenta}}
\newcommand{\restoretargetcolour}[0]{\renewcommand{\ulccol}[0]{RedOrange}}


%%%%%%%%%%%%%%%%%%%%%%%%%%%%%%%%%%%%%%%%%%%%%%%%%%%%%%%%%%%
%	logrel 
%%%%%%%%%%%%%%%%%%%%%%%%%%%%%%%%%%%%%%%%%%%%%%%%%%%%%%%%%%%
\newcommand{\oftype}[1]{\fun{oftype}{#1}}
\newcommand{\oftypes}[1]{\src{oftype\left(#1\right)}}
\newcommand{\oftypet}[1]{\trg{oftype\left(#1\right)}}

\newcommand{\logrelgen}[1]{\ensuremath{\operatorname{\approx}^{#1}}}
\DeclareMathOperator\logrel{\logrelgen{}}
\DeclareMathOperator\bothlogrel{\logrelgen{}}
\newcommand{\underapproxlogrelgen}[1]{\ensuremath{\operatorname{\lesssim}^{#1}}}
\DeclareMathOperator\underlogrel{\underapproxlogrelgen{}}
\newcommand{\overapproxlogrelgen}[1]{\ensuremath{\operatorname{\gtrsim}^{#1}}}
\DeclareMathOperator\overlogrel{\overapproxlogrelgen{}}

\newcommand{\arbsim}{\ensuremath{\triangledown}}%{\ensuremath{\square}} 
\newcommand{\arbsimrel}{\ensuremath{\mathrel{\arbsim}}}
\newcommand{\genlogrel}[0]{\arbsim}
\DeclareMathOperator\anylogrel{\genlogrel{}}

\newcommand{\underlogreln}[1]{\ensuremath{\mathrel{\lesssim_{#1}}}}
\newcommand{\overlogreln}[1]{\ensuremath{\mathrel{\gtrsim_{#1}}}}
\newcommand{\anylogreln}[1]{\ensuremath{\mathrel{\triangledown_{#1}}}}

\newcommand{\genrel}[4]{\ensuremath{\bl{\mc{#1}\left\llbracket\src{#2}\right\rrbracket^{#3}_{#4}}}}
\newcommand{\valrel}[1]{\genrel{V}{#1}{}{\anylogrel}}
\newcommand{\valrelp}[2]{\genrel{V}{#1}{#2}{\anylogrel}}
\newcommand{\contrel}[1]{\genrel{K}{#1}{}{\anylogrel}}
\newcommand{\contrelp}[2]{\genrel{K}{#1}{#2}{\anylogrel}}
\newcommand{\termrel}[1]{\genrel{E}{#1}{}{\anylogrel}}
\newcommand{\termrelp}[2]{\genrel{E}{#1}{#2}{\anylogrel}}
\newcommand{\envrel}[1]{\genrel{G}{#1}{}{\anylogrel}}
\newcommand{\envrelp}[2]{\genrel{G}{#1}{#2}{\anylogrel}}
\newcommand{\tyenvrel}[1]{\genrel{D}{#1}{}{\anylogrel}}

\newcommand{\langsp}[1]{\ensuremath{\mi{#1}}}
\newcommand{\langspfun}[2]{\ensuremath{\langsp{#1}(#2)}}
\newcommand{\W}[0]{\langsp{W}\xspace}

\newcommand{\monotfun}[1]{\fun{\nearrow}{#1}}

\newcommand{\stepsgen}[1]{\ensuremath{\langsp{lev}^{#1}}}
\newcommand{\stepsfungen}[2]{\ensuremath{\langspfun{lev^{#1}}{#2}}}
\newcommand{\steps}[0]{\stepsgen{}}
\newcommand{\stepsfun}[1]{\stepsfungen{}{#1}}

\newcommand{\latergen}[1]{\ensuremath{\triangleright^{#1}}}
\DeclareMathOperator\later{\latergen{}}
\newcommand{\laterfun}[1]{\langspfun{\triangleright}{#1}}

\newcommand{\obswgen}[1]{\ensuremath{\langsp{O}^{#1}}}
\newcommand{\obswfungen}[3]{\langspfun{O^{#1}}{#2}_{#3}}
\newcommand{\obsw}[0]{\obswgen{}}
\newcommand{\obsfun}[2]{\obswfungen{}{#1}{#2}}

\newcommand{\futwgen}[1]{\ensuremath{\sqsupseteq^{#1}}}
\newcommand{\strfutwgen}[1]{\ensuremath{\sqsupset_{\later}^{#1}}}
\DeclareMathOperator\futw{\futwgen{}}
\DeclareMathOperator\futwpub{\futwpubgen{}}
\DeclareMathOperator\strfutw{\strfutwgen{}}

\newcommand{\xto}[1]{\ensuremath{~\mathrel{\xrightarrow{~#1~}}~}}
\newcommand{\Xto}[1]{\ensuremath{~\mathrel{\xRightarrow{~#1~}}~}}



%%%%%%%%%%%%%%%%%%%%%%%%%%%%%%%%%%%%%%%%%%%%%%%%%%%%%%%%%%%
%	backtrans
%%%%%%%%%%%%%%%%%%%%%%%%%%%%%%%%%%%%%%%%%%%%%%%%%%%%%%%%%%%
\newcommand{\psd}[1]{\src{\hat{#1}}}
\newcommand{\emuldv}[1]{\src{EmulDV_{#1}}}
\newcommand{\emtotau}[1]{\fun{repEmul}{#1}}
\newcommand{\srctotrgty}[1]{\fun{srToTr}{#1}}
\newcommand{\uval}[1]{\src{UVal_{#1}}}
\newcommand{\toemul}[1]{\fun{toEmul}{#1}}
\newcommand{\touval}[1]{\fun{toUVal}{#1}}

\newcommand{\myomega}[0]{omega}
\newcommand{\unk}[0]{unk}

\newcommand{\emulate}[2]{\src{emulate_{#1}\left(\trg{#2}\right)}}

\newcommand{\upgrade}[1]{\src{upgrade_{#1}~}}
\newcommand{\downgrade}[1]{\src{downgrade_{#1}~}}
\newcommand{\intag}[1]{\src{in_{#1}~}}
\newcommand{\casetag}[1]{\src{case_{#1}~}}
\newcommand{\inject}[1]{\src{inject_{#1}}}
\newcommand{\extract}[1]{\src{extract_{#1}}}

\newcommand{\upgradepar}[2]{\src{upgrade_{#1}\left(\src{#2}\right)}}
\newcommand{\downgradepar}[2]{\src{downgrade_{#1}\left(\src{#2}\right)}}
\newcommand{\intagpar}[2]{\src{in_{#1}\left(\src{#2}\right)}}
\newcommand{\casetagpar}[2]{\src{case_{#1}\left(\src{#2}\right)}}
\newcommand{\injectpar}[2]{\src{inject_{#1}\left(\src{#2}\right)}}
\newcommand{\extractpar}[2]{\src{extract_{#1}\left(\src{#2}\right)}}


\def\teqaux#1{\vcenter{\hbox{\ooalign{\hfil
       \raise6pt \hbox{\scriptsize{T}}\hfil\cr\hfil
       $=$}}}}
\def\teqa{\mathrel{\mathpalette\teqaux{}}}
\def\nteqa{\mathrel{\mathpalette\teqaux{}\!\!\!/\ }}

\DeclareMathOperator\teq{\ensuremath{\teqa}}
\DeclareMathOperator\nteq{\ensuremath{\nteqa}}

\newcommand{\ift}[0]{\text{ \com{if} }}
\newcommand{\thent}[0]{\text{ \com{then} }}
\newcommand{\andt}[0]{\text{ \com{and} }}
\newcommand{\wheret}[0]{\text{ \com{where} }}

\DeclareMathOperator\relate{\bl{\approx}}
\DeclareMathOperator\relatebeta{\bl{\approx_{\beta}}}
